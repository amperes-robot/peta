% Appendix A
\appendix
\newpage
\section{Appendix: Calculations}
\label{AppendixA}

	\subsection{Rear Wheel Torque}
	From Solidworks' analysis of our CAD models, we have an upper bound $m$ of 5~kg on the assembly's weight. The height $h_G$ of the centre of gravity $G$ off the ground is 0.144~m. The horizontal distance $l_A$ from the back (driven) wheels at $A$ to $G$ is 0.085~m. The horizontal distance $l_B$ from the front support $B$ to $G$ is 0.134~m. From our measurements, the ramp has an incline of 8$^\circ$. The rear motors apply a frictional force $F_F$ on the back wheels, and the front support is estimated to have a coefficient of static friction $\mu$~=~0.2. Applying force and moment balancing equations on a static body,
	
	$$\Sigma F_x = 0:~~ N_A\hat{\imath} + N_B\hat{\imath} - mg\cos{14^\circ}\hat{\imath} = 0$$
	$$\Sigma F_y = 0:~~ F_F\hat{\jmath} - \mu N_B\hat{\jmath}  - mg\sin{14^\circ}\hat{\jmath} = 0$$
	$$\Sigma M = 0:~~ N_Bl_B - N_Al_A + F_Fh_G - \mu N_Bh_G= 0$$
	
	% wolfram alpha: a+b-5*9.8*cos8=0, f-0.2b-5*9.8*sin8=0, b*0.134-a*0.085+f*0.144-0.2*b*0.144=0
	Solving these equations, we obtain 9.7~N force necessary to drive the assembly out of stasis on the ramp, which will be the most difficult part of the course. The wheels have radius 2.8~cm, so a gear ratio of 3:1 will be sufficient to sustain movement with less than half of maximum torque.
	
	\subsection{Actuator Torque}
	Again, from the Solidworks analysis, the actuator arm has a centre of mass 9~cm from the pivot and weighs 0.535~kg when it is loaded with a pet. This requires a torque of 48~Ncm, and consequently a gear ratio of 5:1 will allow the motor to run at half of maximum torque.
	
	\subsection{Zipline Arm Torque}
	The zipline arm has a mass of 0.117~kg which is centred 1.5~cm from the pivot. This is a 1.7~Ncm moment required, which the motor can run at one-tenth of maximum torque without any gearing.
	
	\subsection{Winch Torque}
	The robot has an upper bound of 5~kg on mass. The winch pulley has a radius of 1~cm, which can increase to an upper bound of 2~cm when the belt is wrapped around the spindle. This requires an upper bound of a 100~Ncm moment. A gear ratio of 1:40 is used, allowing the motor to run at just over half of maximum torque.
	
	\subsection{IR Sensing Circuit Value Validation}
		\subsubsection{DC Block}
		Since:
		$$f_{cutoff} = \frac{1}{2\pi RC}$$
		With $R=1k\Omega$ and $C=100nF$
		$$f_{cutoff} = \frac{1}{2\pi(1000)(100\times10^{-9})} = 1.59kHz$$
		This will allow us to block all frequencies below about $1.5kHz$ which will functionally block DC and unwanted low frequencies.
		\begin{figure}[h]
			\centering
			\includegraphics[scale = 0.75]{"Figures/dcBlock".jpg}
			\caption*{DC Block}
		\end{figure}
		
		\subsubsection{Amplifier Calculations}
		$$\frac{V_o}{V_i} =\frac{R_1}{R_2} = \frac{47k\Omega}{10k\omega} = 4.7$$
		This will give us a five times amplification of the signal.
	
		\begin{figure}[h]
			\centering
			\includegraphics[scale = 0.75]{"Figures/amp".jpg}
			\caption*{Amplifying Circuit}
		\end{figure}
		
		\newpage
		\subsubsection{Band Pass Calculations}
		A high pass filter and a low pass filter in series (see diagram below) will form a band pass filter with upper and lower limits determined by:
		$$f_{cutoff} = \frac{1}{2\pi RC}$$
		We want both cutoff frequencies to be $10kHz$ so that the band around the desired frequency is as narrow as possible. Therefore, the capacitors and resistors used in each section of the band pass filter will be the same. \newline \\
		Using $R = 3.3k\Omega$ and $C = 4.7nF$ we get:
		$$f_{cutoff} = \frac{1}{2\pi(3700)(4.7\times10^{-9})} = 10.26kHz$$
		as desired.
		
		\begin{figure}[h]
			\centering
			\includegraphics[scale = 0.75]{"Figures/bandPass".jpg}
			\caption*{Band Pass Filter}
		\end{figure}
		
		\newpage
		\subsubsection{Peak Detector Calculations}
		So that the input to the TINAH is a non-oscillatory analog signal, we will need a peak detector. This will be a rectifying circuit consisting of a resistor and a capacitor. Since the charge/discharge time of an RC circuit is
		$$\tau = RC$$
		we need to pick these values so that the charge time is as short as possible and the discharge time is long enough to compensate for the voltage drop due to the oscillation of the IR signal. We chose $R = 33k\Omega$ and $C = 100 nF$ so that $\tau = 33000(100\times 10^(-9)) = 3.3ms$ so that the discharge time would be longer than the period of the wave, keeping voltage relatively constant, and short enough that IR following will be functional.
		
		\begin{figure}[h]
			\centering
			\includegraphics[scale = 0.75]{"Figures/peakDetector".jpg}
			\caption*{Peak Detector}
		\end{figure}
\newpage
	
\section{Appendix: Circuit Schematics}
\label{AppendixB}
\begin{figure}[h]
	\centering
	\includegraphics[scale = 0.5, angle = 90]{"Figures/SchematicA_HBridge".png}
	\caption*{Schematic A: H-bridge circuitry for controlling motors for driving and arm actuation}
	\label{sch:A}
\end{figure}
\newpage
\begin{figure}[ht]
	\centering
	\includegraphics[scale = 0.5, angle = 90]{"Figures/Schematic_B_QRD".png}
	\caption*{Schematic B: QRD1114 circuitry for tape following}
	\label{sch:B}
\end{figure}

Left QRD1114 is for general tape following; right QRD1114 is for detecting tape leading to pet.

\newpage
\begin{figure}[ht]
	\centering
	\includegraphics[scale = 1, angle = 90]{"Figures/Schematic_C-IR".png}
	\caption*{Schemtic C: Circuitry for 10kHz IR beacon detector}
	\label{sch:C}
\end{figure}

\newpage
\begin{figure}[ht]
	\centering
	\includegraphics[scale = 0.5, angle = 90]{"Figures/Schematic_D-winch".png}
	\caption*{Schematic D: circuitry for controlling winch}
	\label{sch:D}
\end{figure}

\begin{figure}[ht]
	\centering
	\includegraphics[scale = .5, angle = 90]{"Figures/Schematic_E-petpickup".png}
	\caption*{Schematic E: circuit for microswitch detecting pet pick up}
	\label{sch:E}
\end{figure}
\FloatBarrier

\section{Appendix: Software Modes}
\label{AppendixC}

\begin{table}[h]
	\caption{Table of Software Modes}
	\centering
	\begin{tabular}{ | c | p{5in} | }
		\hline
		\textbf{Mode} & \textbf{Description} \\ \hline
		Main Menu & This is the mode automatically entered when the board boots. The knobs are used to cycle between menu options. Additionally, any mode can be canceled to return to the Main Menu. \\ \hline
		Options Menu & Allows runtime parameters to be modified using the second knob. \\ \hline
		Segment Select & Starts the selected mode from a list containing modes during program execution. \\ \hline
		Course Begin & Transitions to the Follow/Retrieval mode. \\ \hline
		Follow/Retrieval & Follows the tape and picks up the first three animals. Upon reaching the end of the tape (once all horizontal marks have been detected), switches to the Beacon Homing mode. \\ \hline
		Beacon Homing & Using the two forward IR sensors, navigates towards the IR beacon. Picks up the elevated pet on the rafters by measuring distance from the end of the tape. Switches to Rubble Excavation upon IR sensor intensity reaching a threshold to be determined empirically. \\ \hline
		Rubble Excavation & Turns 90 degrees and brushes off the top layer of foam using the arm. Attempts to find and retrieve the pet buried in the foam, and upon retrieving it (or on approaching the 2 minute time limit), switches to Zipline/Return. \\ \hline
		Zipline/Return &
		Uses the side IR sensor to align the robot with the beacon, raises the zipline arm, and cranks the winch up to return to the Safe Area. Returns to Main Menu afterwards.
		\\ \hline
	\end{tabular}
	\label{table:Software Modes}
\end{table}
