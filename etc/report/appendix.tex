% Appendix A
\newpage
\appendix
\section{Appendix: Calculations}
\label{AppendixA}

	\subsection{Rear Wheel Torque}
	From Solidworks' analysis of our CAD models, we have an upper bound $m$ of 5~kg on the assembly's weight. The height $h_G$ of the centre of gravity $G$ off the ground is 0.144~m. The horizontal distance $l_A$ from the back (driven) wheels at $A$ to $G$ is 0.085~m. The horizontal distance $l_B$ from the front support at $B$ to $G$ is 0.134~m. From our measurements, the ramp has an incline of 14$^\circ$. Applying force and moment balancing equations on a static body,
	
	$$N_A$$
	
\newpage	
\section{Appendix: Circuit Schematics}
\label{appendixB}
\begin{figure}[h]
	\centering
	\includegraphics[scale = 0.5, angle = 90]{"Figures/SchematicA_HBridge".png}
	\caption*{Schematic A: H-bridge circuitry for controlling motors for driving and arm actuation}
	\label{sch:A}
\end{figure}
\newpage
\begin{figure}[h]
	\centering
	\includegraphics[scale = 0.5, angle = 90]{"Figures/Schematic_B_QRD".png}
	\caption*{Schematic B: QRD1114 circuitry for tape following}
	\label{sch:B}
\end{figure}

Left QRD1114 is for general tape following; right QRD1114 is for detecting tape leading to pet.

\begin{figure}[h]
	\centering
	\includegraphics[scale = 1, angle = 90]{"Figures/Schematic_C-IR".png}
	\caption*{Schematic C: Circuitry for 10kHz IR beacon detector}
	\label{sch:C}
\end{figure}

\begin{figure}[h]
	\centering
	\includegraphics[scale = 0.5, angle = 90]{"Figures/Schematic_D-winch".png}
	\caption*{Schematic D: circuitry for controlling winch}
	\label{sch:D}
\end{figure}

\begin{figure}[h]
	\centering
	\includegraphics[scale = 0.5, angle = 90]{"Figures/Schematic_E-petpickup".png}
	\caption*{Schematic E: circuit for microswitch detecting pet pick up}
	\label{sch:E}
\end{figure}