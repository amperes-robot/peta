% Appendix A
\appendix
\newpage
\section{Appendix: Calculations}
\label{AppendixA}

	\subsection{Rear Wheel Torque}
	From Solidworks' analysis of our CAD models, we have an upper bound $m$ of 5~kg on the assembly's weight. The height $h_G$ of the centre of gravity $G$ off the ground is 0.144~m. The horizontal distance $l_A$ from the back (driven) wheels at $A$ to $G$ is 0.085~m. The horizontal distance $l_B$ from the front support $B$ to $G$ is 0.134~m. From our measurements, the ramp has an incline of 8$^\circ$. The rear motors apply a frictional force $F_F$ on the back wheels, and the front support is estimated to have a coefficient of static friction $\mu$~=~0.2. Applying force and moment balancing equations on a static body,
	
	$$\Sigma F_x = 0:~~ N_A\hat{\imath} + N_B\hat{\imath} - mg\cos{14^\circ}\hat{\imath} = 0$$
	$$\Sigma F_y = 0:~~ F_F\hat{\jmath} - \mu N_B\hat{\jmath}  - mg\sin{14^\circ}\hat{\jmath} = 0$$
	$$\Sigma M = 0:~~ N_Bl_B - N_Al_A + F_Fh_G - \mu N_Bh_G= 0$$
	
	% wolfram alpha: a+b-5*9.8*cos8=0, f-0.2b-5*9.8*sin8=0, b*0.134-a*0.085+f*0.144-0.2*b*0.144=0
	Solving these equations, we obtain 9.7~N force necessary to drive the assembly out of stasis on the ramp, which will be the most difficult part of the course. The wheels have radius 2.8~cm, so a gear ratio of 3:1 will be sufficient to sustain movement with less than half of maximum torque.
	
	\subsection{Actuator Torque}
	Again, from the Solidworks analysis, the actuator arm has a centre of mass 9~cm from the pivot and weighs 0.535~kg when it is loaded with a pet. This requires a torque of 48~Ncm, and consequently a gear ratio of 5:1 will allow the motor to run at half of maximum torque.
	
	\subsection{Zipline Arm Torque}
	The zipline arm has a mass of 0.117~kg which is centred 1.5~cm from the pivot. This is a 1.7~Ncm moment required, which the motor can run at one-tenth of maximum torque without any gearing.
	
\newpage	
\section{Appendix: Circuit Schematics}
\label{AppendixB}
\begin{figure}[h]
	\centering
	\includegraphics[scale = 0.5, angle = 90]{"Figures/SchematicA_HBridge".png}
	\caption*{Schematic A: H-bridge circuitry for controlling motors for driving and arm actuation}
	\label{sch:A}
\end{figure}
\newpage
\begin{figure}[h]
	\centering
	\includegraphics[scale = 0.5, angle = 90]{"Figures/Schematic_B_QRD".png}
	\caption*{Schematic B: QRD1114 circuitry for tape following}
	\label{sch:B}
\end{figure}

Left QRD1114 is for general tape following; right QRD1114 is for detecting tape leading to pet.

\begin{figure}[h]
	\centering
	\includegraphics[scale = 1, angle = 90]{"Figures/Schematic_C-IR".png}
	\caption*{Schematic C: Circuitry for 10kHz IR beacon detector}
	\label{sch:C}
\end{figure}

\begin{figure}[h]
	\centering
	\includegraphics[scale = 0.5, angle = 90]{"Figures/Schematic_D-winch".png}
	\caption*{Schematic D: circuitry for controlling winch}
	\label{sch:D}
\end{figure}

\begin{figure}[h]
	\centering
	\includegraphics[scale = 0.5, angle = 90]{"Figures/Schematic_E-petpickup".png}
	\caption*{Schematic E: circuit for microswitch detecting pet pick up}
	\label{sch:E}
\end{figure}