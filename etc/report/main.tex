%----------------------------------------------------------------------------------------
%	PACKAGES AND OTHER DOCUMENT CONFIGURATIONS
%----------------------------------------------------------------------------------------

\documentclass[11pt, oneside]{article} % The default font size and one-sided printing (no margin offsets)
\usepackage{graphicx}
\usepackage{caption}
\usepackage{subcaption}
\usepackage{placeins}
\graphicspath{{Figures/}}
\usepackage[square, numbers, comma, sort&compress]{natbib} % Use the natbib reference package - read up on this to edit the reference style; if you want text (e.g. Smith et al., 2012) for the in-text references (instead of numbers), remove 'numbers' 
\usepackage[parfill]{parskip}
\usepackage[margin=1in]{geometry}
\usepackage{tabularx}
\usepackage{array}
\usepackage{hyperref}
\usepackage{xcolor}
\definecolor{link}{rgb}{0, 0, 0.5}
\hypersetup{
	colorlinks,
	citecolor=link,
	filecolor=link,
	linkcolor=link,
	urlcolor=link
}
\newcolumntype{P}[1]{>{\centering\arraybackslash}p{#1}}
\renewcommand{\arraystretch}{1.5}
\newcommand{\robotname}{AutoMated PEt REscue System (A.M.P.E.R.E.S)}

\begin{document}

% Define the page headers using the FancyHdr package and set up for one-sided printing

\newcommand{\HRule}{\rule{\linewidth}{0.5mm}} % New command to make the lines in the title page

%----------------------------------------------------------------------------------------
%	TITLE PAGE
%----------------------------------------------------------------------------------------

\begin{titlepage}
\begin{center}

\textsc{\LARGE ENPH 253 Design Proposal}\\[1.5cm]

\HRule \\[0.4cm] % Horizontal line
{\huge \bfseries \robotname}\\[0.07cm]
\HRule \\[1.5cm] % Horizontal line

\begin{minipage}[t]{0.4\textwidth}
\begin{flushleft} \large
\textbf{Team Members} \\
{Scott Fjordbotten \newline Anne Lim \newline Johnson Liu \newline Kevin Zhang} % Author names
\end{flushleft}
\end{minipage}
\begin{minipage}[t]{0.4\textwidth}
\begin{flushright} \large
\textbf{Instructors} \\
{Dr. Andre Marziali \\ Dr. Jon Nakane \\ Mr. Bernhard Zender \\ Ms. Pamela Rogalski} % Instructors names
\end{flushright}
\end{minipage}\\[3cm]
 
 
{\large June 2015}\\[4cm]
%\includegraphics{Logo} % University/department logo - uncomment to place it
 
\vfill
\end{center}

\end{titlepage}


%----------------------------------------------------------------------------------------
%	Executive Summary
%----------------------------------------------------------------------------------------
\section*{Executive Summary}
This document contains the proposed specifications of a robot designed to complete the 2015 ENPH 253 Pet-Rescue Bots challenge. The robot should be able to navigate through an 8' by 8' course by following a tape path and infrared beacon, collect ``Pets'', which are Beanie toys of height 6" with magnets attached to them and return to the start area. Our robot will transport itself and the toys to a "start area" via the zip line of height 19", which spans the course.

The robot must be able to go through a doorway which is 14" wide and 18" tall. Most pets will be placed 8" to the right of the tape path, one will be placed on a platform 6" above the playing surface, and the last pet will be located in a container and covered with Styrofoam bricks. All will be unattached from the playing surface.

Our robot will be controlled by a TINAH Board using a ATMega128 processor. The proposed design for the robot has an upper bound of 5~kg and a lower bound of 3~kg mass. The design has a length of 15.74", a width of 11.61", and a height of 14.5" when fully retracted.

\begin{figure*}[h]
	\centering
	\includegraphics{"Figures/course".jpg}
	\caption*{The Competition Course, Pets marked P1-6}
\end{figure*}

\newpage

%----------------------------------------------------------------------------------------
%	LIST OF CONTENTS/FIGURES/TABLES PAGES
%----------------------------------------------------------------------------------------

\tableofcontents % Write out the Table of Contents
\newpage
\listoffigures % Write out the List of Figures
\listoftables % Write out the List of Tables

%----------------------------------------------------------------------------------------

\newpage
%	Document Content
%---------------------------------------------------------------------------------------

%--------------------------------------------
%PREFACE
\section{Preface}

This report was written in collaboration by Kevin Zhang, Scott Fjordbotten, Johnson Liu, and Anne Lim, in the hope of identifying and addressing potential issues with the design. Work was begun on this document in the beginning of June 2015. We would like to thank the teaching assistants and professors of ENPH 253 for their support and advice throughout.

In the process of preparing this proposal, all major design decisions were agreed upon by unanimous consent. Listed below is a summary of the work done by each team member on this report.

\begin{center}
	\begin{tabular}{ l l }
		FJORDBOTTEN & Typesetting, Mechanical Design \\
		LIM & Letter of Transmittal, Editing \\
		LIU & Electronics and Sensors \\
		ZHANG & Software, Editing \\
	\end{tabular}
\end{center}

%--------------------------------------------
%Overview of Basic Strategy
\section{Overview of Basic Strategy}

The robot will contain a meshed in area at the front for carrying the payloads. Pets will be picked up by a plastic arm with a steel bracket on the end which will move along a circular path in the plane parallel to the front of the robot to magnetically attach and collect the pets. Below the steel bracket will be attached a hinged aluminium plate that will be sandwiched between the steel plate and any attached magnet, triggering a micro-switch when a magnet is attached. A rod fixed to the chassis will be located at the end of the arm's path such that the load will be sheared off the metallic plate in order to land in the meshed area.

The robot will start following the tape in the starting location and follow it using a single QRD sensor connected to an analog input on the TINAH board and a proportional-integral-derivative controller. Upon detecting tape markers perpendicular to the main tape path using a side QRD connected to a LM311 comparator and a digital input, the retrieval arm will then be lowered until the attachment signal is received from the micro-switch. At this point, the arm will retract until the load is sheared off using the fixed rod, which can be detected by the falling signal from the micro-switch. The pet located in the middle of the path will magnetically attach to a steel bracket at the front of the robot and will remain attached to this bracket for the duration of the heat. The pet located on the elevated rafters will be located using the IR beacon and the rotary encoders on the wheels to determine the distance travelled since the end of the tape.

After collecting the elevated pet, the robot will continue to follow the IR beacon until the intensity from the beacon reaches a certain threshold, to be determined empirically. Based on encoder data, the robot will pivot 90 degrees to position the retrieval arm along the edge of the box. The retrieval arm will be used to push the Styrofoam rubble aside and retrieve the last pet.

At this point, the robot should be positioned below and facing away from the zipline. An arm located on the left side of the robot opposite the pet retrieval arm and containing a magnetically attached zipline trolley will swing in the plane parallel to the side of the robot to attach the trolley to the zipline. A sensor will be used to determine when the trolley is on the zipline (sensor type to be determined). A winch in the centre of the robot and attached to the bottom side of the trolley will simultaneously detach the trolley from the zipline arm and lift the robot off the ground. Once off the ground, the robot will roll down the zipline to the Safe Area by virtue of gravity.

\begin{figure}[h]
	\centering
	\includegraphics[scale=0.5]{"Figures/Software State Diag".png}
	\caption[Software State Diagram]{Software State Diagram}
	\label{fig:SoftStates}
\end{figure}

%--------------------------------------------
%Mechanical Design
\section{Mechanical Design}

Our robot will consist of three main mechanical sub-assemblies: The Chassis, The Retrieval Arm, and The Zipline Apparatus. These assemblies have been designed such that they are independent modules that can be built and tested at a basic level in parallel. This design section has been broken into sections to show each element individually before being combined into the complete version of the robot.

	\subsection{The Chassis}

	The Chassis is the base for the robot. The Chassis consists of the Chassis Plate and two supports for the rear drive motors and axles. The Chassis Plate includes two vertical sections at the rear that support the components of the Board Storage Space. These sections will be discussed in more detail in the 'Board Storage Space' subsection. Circular holes have been integrated in the front portion of the chassis plate to allow for wire routing to the tape following and tape detection QRD1114 reflectance sensors. The Chassis Plate and motor supports will be manufactured from 20 gauge aluminium to limit the overall weight of the robot. Additional brackets may be fabricated to reinforce the edge flange joints if necessary. The complete Chassis and Board Storage module is shown in \autoref{fig:chassisAssy}.

	\begin{figure}[h]
		\centering
		\includegraphics[scale=0.5]{"Figures/Chassis Body".jpg}
		\caption[Chassis]{The Chassis}
		\label{fig:1}
	\end{figure}

		\subsubsection{Pet 4 Pickup Arm}
		The fourth pet (in the middle of the path) will be picked up statically. To accomplish this there will be a steel bracket mounted to the front of the chassis via a acrylic arm (See \autoref{fig:front arm}). The steel bracket will be at the height of the pets head and reinforced with magnets to ensure the pet does not fall off in the IR and zipline phases of the course.
	
		\subsubsection{The Sled}
		Rather than building structures to support wheels at the front of the robot, we will construct a sled and attach it to the front of the chassis via two axle mounted brackets (See \autoref{fig:front sled}). The sled will be free to pivot about the mounting axles to minimize the chance of the sled catching on something and forcing the robot off the desired trajectory. The sled will also have a support (not shown) and hole for the tape following QRD. Both the sled and its mounting brackets will be constructed from 20 gauge aluminum.
		
		\subsubsection{Board Storage Space}

		The Board Storage Space is integrated into the rear of the chassis. The vertical sections of the chassis contain slots and tabs to support the three shelves in the Board Storage Space (See \autoref{fig:3}). The shelves will be 3.4" deep and 7.8" long and will have a flange at the front to prevent boards from sliding off the shelves. The two bottom shelves will house circuit boards; once the bottom two shelves have been loaded with boards, the back plate will be screwed on to secure the shelves. The top shelf will support the TINAH in conjunction with the back plate. Two mounting features for the TINAH are on the front of the top shelf while the remaining two are on the top of the back plate.
		
		\newpage
		\begin{figure}[ht]
			\centering
			\begin{minipage}[t]{.5\textwidth}
				\centering
				\includegraphics[scale=0.5]{"Figures/Board Storage Space".jpg}
				\caption[Board Storage Space]{Board Storage Space}
				\label{fig:2}
			\end{minipage}%
			\begin{minipage}[t]{.5\textwidth}
				\centering
				\includegraphics[scale=0.5]{"Figures/chassis back features".jpg}
				\caption[Board Storage Support Features]{Board Storage Support Features}
				\label{fig:3}
			\end{minipage}
		\end{figure}
		
		\begin{figure}[!ht]
			\begin{minipage}[t]{.5\textwidth}
				\centering
				\includegraphics[scale=0.75]{"Figures/frontSled".jpg}
				\caption[The Sled]{The font sled}
				\label{fig:front sled}
			\end{minipage}%
			\begin{minipage}[t]{.75\textwidth}
				\centering
				\includegraphics[scale=1]{"Figures/frontArm".jpg}
				\caption[Front Arm]{The front arm for statically \\ collecting pet 4}
				\label{fig:front arm}
			\end{minipage}
		\end{figure}
		
		\begin{figure}[ht]
			\centering
			\includegraphics[scale=0.5]{"Figures/chassisAssy".jpg}
			\caption[Complete Chassis and Board Storage Module]{Complete Chassis and Board Storage Module}
			\label{fig:chassisAssy}
		\end{figure}
	
	\newpage	
	\subsection{The Retrieval Arm}
	The retrieval arm consists of two main section: the supporting body (\autoref{fig:armBody}) and the arm end (\autoref{fig:armEnd}). The supporting body is constructed out of 20 gauge aluminum. It houses the arm sections, gears and pet knock off tool (\autoref{fig:knockOffTool})and will attach to the chassis via pop rivet. There will be two arm sections, one strait and one that will double as the large driving gear. Both arms will be cut out of acrylic(\autoref{fig:arm}). The gear ratio between the driving DC motor and the arm gear will be 5:1 as determined in \autoref{AppendixA}. The knock of tool will be made of 3.2mm diameter round steel and positioned so that pets will be knocked off the arm above the chassis. The length and center of rotation of the arm were chosen such that the arm is the same distance from the chassis center(eight inches) when the pick up surface is at both six and twelve inches from the playing surface so that both arms can be used to pick up the pets placed on the surface and the pet that is elevated.
	
		\subsubsection{Retrieval Arm End}
		The arm end will consist of a body made of acrylic, L-brackets made of steel to magnetically attach to pets, and aluminum sheets that will be pushed into micro-switches when a pet is picked up(\autoref{fig:armEnd}). There will be two such L-bracket assemblies. The smaller assembly, facing the front of the robot, will be used to pick up pets 1-4 and 5. This arm will align with the knock off tool to deposit pets in the chassis and the aluminum sheet will pivot on a hinge. The larger assembly, facing the rear of the robot, is at the bottom of the body and will be used to collect pet 6 from the box of foam. This assembly will be reinforced with magnets and the 6th pet will remain on the bracket for the remainder of the course. The aluminum sheet will hang below the L-bracket.
		
	\begin{figure}[h]
		\centering
		\begin{subfigure}[b]{0.4\textwidth}
			\centering
			\includegraphics[scale=0.5]{"Figures/armBodyFront".jpg}
			\caption{Front View}
		\end{subfigure}
		~
		\begin{subfigure}[b]{0.5\textwidth}
			\centering
			\includegraphics[scale=0.5]{"Figures/armBodyRear".jpg}
			\caption{Rear View}
		\end{subfigure}
		\caption[Retrieval Arm Module]{Retrieval Arm Module}
		\label{fig:armBody}
	\end{figure}
		
	\begin{figure}[h]
		\centering
		\begin{minipage}[t]{.5\textwidth}
			\centering
			\includegraphics[scale = 0.4]{"Figures/armEnd".jpg}
			\caption[Retrieval Arm End]{Retrieval Arm End}
			\label{fig:armEnd}
		\end{minipage}%
		\begin{minipage}[t]{.5\textwidth}
			\centering
			\includegraphics[scale=0.4]{"Figures/armArms".jpg}
			\caption[Retrieval Arm Arms]{Retrieval Arm Arms}
			\label{fig:arm}
		\end{minipage}
	\end{figure}
	
	\begin{figure}[!h]
		\centering
		\includegraphics[scale=0.4]{"Figures/knockOffTool".jpg}
		\caption[Knock Off Tool]{Knock Off Tool}
		\label{fig:knockOffTool}
	\end{figure}
	
	\newpage
	\subsection{The Zipline Apparatus}
	The zipline apparatus is contained in a body similar to the pet retrieval arm body. This body houses the motors for the zipline delivery system and the winch system, the gears for these systems the zipline delivery arm and the zipline trolley (\autoref{fig:zipModule}). The body will be made from 20 gauge aluminum and the gears will be made ot of acrylic.
	
		\subsubsection{Zipline Delivery System}
		The zipline delivery system consists of a guide that swings the zipline trolley into place. The guide has flanges around all sides to prevent the trolley from falling out before the trolley is deployed. The trolley frame and guide will be made of aluminium and will have small magnets attached to prevent premature separation.
		
		\subsubsection{The Zipline Trolley}
		The zipline trolley consists of a frame, winch attachment axle and roller (\autoref{fig:zipTrolley}). The frame was designed with flanges all the way around to improve rigidity. A finite element analysis was carried out in Solidworks. The results showed stresses much below the plastic limit and deflections of a fraction of a millimetre (See \autoref{fig:zipFEA}). Although the validity of the FEA is questionable, the results would have to be off by two orders of magnitude for deflections to be concerning so we are confident our design is sufficient. The winch belt will be attached to the trolley via a pin at the bottom of the frame. The pin bracket will be manufactured from 20~gauge aluminium. The top of trolley supports the roller which will sit on the zipline, allowing the robot to use the force of gravity to slide to safety once lifted by the winch. The roller will be lathed out of Ultra High Molecular Weight Polyethylene.
		
		\subsubsection{The Winch System}
		The robot will be lifted off the ground by the winch system, which attaches to the zipline trolley. The winch will be driven by a worm gear system with a 1:40 gear ratio to ensure ample torque and prevent back-driving (\autoref{fig:winch}). The winch will be connected to the trolley by a 1.5" wide material that is to be determined.
		
		\begin{figure}[h]
			\centering
			\begin{minipage}[t]{.5\textwidth}
				\centering
				\includegraphics[scale = 0.4]{"Figures/ziplineModule".jpg}
				\caption[Zipline Module]{Zipline Module}
				\label{fig:zipModule}
			\end{minipage}%
			\begin{minipage}[t]{.5\textwidth}
				\centering
				\includegraphics[scale=0.4]{"Figures/zipTrolley".jpg}
				\caption[Zipline Trolley]{Zipline Trolley}
				\label{fig:zipTrolley}
			\end{minipage}
		\end{figure}	
		
		\begin{figure}[h]
			\centering
			\begin{minipage}[t]{.5\textwidth}
				\centering
				\includegraphics[scale = 0.4]{"Figures/ziplineZipperFEA".jpg}
				\caption[Zipline Trolley FEA]{Zipline Trolley FEA}
				\label{fig:zipFEA}
			\end{minipage}%
			\begin{minipage}[t]{.5\textwidth}
				\centering
				\includegraphics[scale=0.4]{"Figures/winch".jpg}
				\caption[WinchSystem]{Winch System}
				\label{fig:winch}
			\end{minipage}
		\end{figure}
	
	\newpage
	\subsection{The Robot Assembly}
	When the sub-assemblies have been constructed and tested, they will be combined to form our robot (\autoref{fig:RobotAssy}). The sub-assemblies will be pop riveted to the Chassis Plate.
	
	\begin{figure}[h]
		\centering
		\includegraphics[scale=0.7]{"Figures/robotAssy".jpg}
		\caption[Robot Assembly]{Robot Assembly}
		\label{fig:RobotAssy}
	\end{figure}
	
	\newpage
	
%--------------------------------------------
%Drive and Actuator Systems
\section{Drive and Actuator Systems}
The drive and actuator systems will consist of five Barber-Coleman geared DC motors. Four of the five will be controlled by H-Bridge circuitry while the winch motor will run unidirectionally on a single MOSFET.

	\subsection{Drive System}
	The drive system will consist of two DC motors connected to the drive wheel axles through encoders. Both drive motors will be controlled by H-bridge circuits. The gear ratio between the wheels and the motors will be 3:1 as calculated in \autoref{AppendixA}. All gears will be located in the gap between the vertical section of the Chassis Plate and the inside edge of the wheels. See \autoref{fig:chassisAssy} for the location of the drive system.
	
	\subsection{Retrieval Arm Actuation}
	The retrieval arm will be actuated by a single, H-bridge controlled DC motor. The gear ratio between the arm and the motor will be 5:1 as calculated in \autoref{AppendixA}. The gears for the retrieval arm will be contained within the retrieval arm body.
	
	\subsection{Zipline Apparatus Actuation}
	The zipline apparatus will consist of one actuator for the delivery system and another for the winch. See \autoref{fig:zipModule} for the location of these motors.
		
		\subsubsection{Zipline Delivery System Actuation}
		The zipline delivery system will be actuated by a single DC motor controlled by an H-bridge. The motor will be mounted at the top of the body of the zipline system. The gear ratio between the delivery arm and the motor will be 1:1 as calculated in \autoref{AppendixA}.
		
		\subsubsection{Zipline Winch Actuation}
		The winch will be actuated by a unidirectional DC motor. This motor will be located in the bottom of the zipline body and will be controlled by a single MOSFET. The motor will be connected to the winch by a worm gear with a 40:1 ratio to ensure sufficient torque and to prevent back-driving (as determined in \autoref{AppendixA}).

%--------------------------------------------
%Sensor System

\section{Sensor System}
	
	\subsection{Tape Following}
	
	There will be one analog QRD1114 phototransistor at the front of the robot for reading reflectance values to detect the tape on the playing surface. It will work in conjunction with the wheel encoders to ensure efficient tape following. Another QRD will be placed on the side of the robot for detecting the pet marker tape.
	
	\subsection{IR Detection}
	Two front-facing QSD124 IR photodiodes for detecting the 10kHz rescue beacon will be used to triangulate the distance and angle of the beacon. An additional photodiode will be left-facing in order to align the robot with the beacon at the end of the run.
	
	\subsection{Wheel Encoders}
	
	The wheel encoders will be mounted on the drive train, and will be used to calculate position and also attempt to recover from the loss of tape.
	
	\subsection{Microswitches}
	
	Two microswitches attached to the actuator arm's will be triggered when a pet is attached to the arm. This will be used to determine whether the pet has been picked up and when it is removed by the shearing pole. An additional microswitch may be used for zipline detection, the type of sensor used for zipline detection has not be finalized.
	
%--------------------------------------------
%Electrical Design

\section{Electrical Design}

Wires will be routed along the centre of the robot, up along the front of the board storage space and to the shelf where the circuitry will reside. Wire routes are shown in red in \autoref{fig:wire routing}). Microswitch wires from the retrieval arm will run along the arms, through the retrieval arm body and to the connection points on the TINAH. Motor wires for the zipline sub-assembly will be kept within the zipline body whenever possible so that the aluminium body doubles as an EMR shield. Motor control circuits will be on the lower shelf of the board storage space while IR detection circuits will be on the top shelf. Shielded wires will be used for sensitive signal wires to further protect against noise from power wires and circuits.

\begin{figure}[h]
	\centering
	\includegraphics[scale = 0.75]{"Figures/wireRouting".jpg}
	\caption[Wire Routing]{Wire routes shown in red}
	\label{fig:wire routing}
\end{figure}

	\newpage
	\subsection{Drive}
	
	There will be two motors for driving, each powering one wheel using differential steering. Each motor will be connected to a motor output on TINAH via an external H-bridge with a comparator attached. Similarly, another H-bridge is used for actuating the arm for picking up pets, shown in \hyperref[sch:A]{Schematic A}.
	
	\subsection{Phototransistors}
	
	The QRD1114 phototransistor for tape following will be connected directly to an analog input pin on TINAH. Another QRD1114 will be placed on the side of the robot and connected to a digital input on TINAH with an adjustable comparator for detecting the pet marker tape. This is shown in \hyperref[sch:B]{Schematic B}.
	
	\subsection{Photodiodes}
	
	\hyperref[sch:C]{Schematic C} shows the circuit for the three identical IR detectors used for detecting the IR beacons during the competition.
	
	\subsection{Winch}
	
	One N-type MOSFET drives the motor for the winch. Since it does not need to be reversible, a full H-bridge is not needed.
	
	\subsection{Pickup Microswitches}
	
	The microswitches connected to the lower and upper plates of the retrieval arm are connected as shown in \hyperref[sch:E]{Schematic D}
	
	\begin{table}[h]
		\caption{Table of TINAH Pin Connections}
		\centering
		\begin{tabular}[t]{|P{0.5in}|p{2in}|}
			\hline
			\textbf{Digital Pin} & \textbf{Input/Output} \\
			\hline
			0 & Pet Marker QRD \\
			\hline
			1 & Pickup Microswitch Lower \\
			\hline
			2 & Pickup Microswitch Upper \\
			\hline
			3 & Winch Enable \\
			\hline
			4 & Wheel Encoder Left \\
			\hline
			5 & Wheel Encoder Right\\
			\hline
			6 & Zipline Detection\\
			\hline
		\end{tabular}
		\quad
		\begin{tabular}[t]{|P{0.5in}|p{2in}|}
			\hline
			\textbf{Analog Pin} & \textbf{Input/Output} \\
			\hline
			0 & IR Sensor Left \\
			\hline
			1 & IR Sensor Right \\
			\hline
			2 & Tape Follower QRD \\
			\hline
			3 & IR Sensor Side \\
			\hline
			\multicolumn{2}{c}{ } \\
			\hline
			\textbf{Motor Pin} & \textbf{Output}\\
			\hline
			0 & Drive Motor Left \\
			\hline
			1 & Drive Motor Right \\
			\hline
			2 & Retrieval Arm Motor \\
			\hline
			3 & Zipline Arm Motor \\
			\hline
		\end{tabular}
		\label{table:TINAHpins}
	\end{table}
	
	\begin{table}[h]
		\caption{Table of PCB Information}
		\centering
		\begin{tabular}[t]{ | P{0.7in} | p{1.5in} | P{0.7in} | p{1.2in} | P{1in} | }
			\hline
			\textbf{PCB Number} & \textbf{Purpose} & \textbf{Size (mm)} & \textbf{Components Connected} & \textbf{Rails Needed} \\
			\hline
			1 & H-bridges: Drive Motors Left and Right, Retrieval Arm Motor, Zipline Arm Motor & 180 x 70 & TINAH, 4 DC motors & GND, 5V, 15V\\
			\hline
			2 & IR detection circuitry & 90 x 70 & TINAH, 2 photo-diodes & GND, 5V, $\pm$ 9V\\
			\hline
		\end{tabular}
		\label{table:PCBInfo}
	\end{table}
	
%--------------------------------------------
%Softwate and Code Algorithms
\newpage
\section{Software and Code Algorithms}	
The software will operate statefully, with one main control loop that handles the primary strategy and execution. At the same time, a 10kHz timer interrupt will run in parallel which will handle the time-sensitive operation of polling input pins.
The main control loop will call upon a variety of different modes that describe the current operation at any given point in the program's execution. A summary of the different modes and their transitions is described in \autoref{fig:4} below.

\begin{figure}[h]
	\centering
	\includegraphics[scale=0.5]{"Figures/Software Modes".jpg}
	\caption[Software Modes]{Software Modes}
	\label{fig:4}
\end{figure}

A more detailed description of the modes is shown in \autoref{AppendixC}.
	
	\subsection{I/O}
	
	On the 10kHz timer interrupt will be attached a procedure that handles pin inputs. Because the standard \texttt{analogRead} function blocks until the ADC conversion finishes (approximately a 1000-cycle process), carrying out the conversion in parallel can significantly decrease loop latency. Additionally, polling digital pin inputs on the timer interrupt can provide a better guarantee that changes in digital inputs will not be missed because of latency in the main loop.
	
	Inputs to digital pins 0 to 3 were chosen because of their ability to trigger external interrupts. As a result, we will not poll these pins on the timer interrupt.
	
	\subsection{Positional Correction}
	
	The encoders, tape-seeking QRD, and side QRD make up the position correction mechanism. Using all of the below methods, we can accurately navigate the course.
	
		\subsubsection{Tape Following}
		
		Tape following is the procedure of using a positional feedback model to adjust velocity based on a physical track. The analog tape-seeking QRD is read into the following formula which calculates an output value $\delta\theta$ based on the QRD reading $Q$:
		$$-\delta\theta = G_P (Q - Q_T) + G_I \int_{0}^{t} (Q - Q_T) \delta t + G_D \frac{\delta (Q - Q_T)}{\delta t}$$
		where the gains $G_P$, $G_I$, $G_D$ are to be empirically determined, and $Q_T$ is the desired value of $Q$ (in this case, the half-way point between the black and white readings). The robot will then attempt to follow the boundary between the tape and the non-taped area.
		
		\subsubsection{Dead Reckoning}
		
		Dead reckoning is the process of calculating position based on an initial condition and velocity data over the course of travel. From encoder data, we can derive estimates for the change in position and angle with numerical integration. The number of encoder ticks corresponds proportionally to the instantaneous left and right wheel velocities $v_l$ and $v_d$. With a distance of $l$ between the two wheels, the instantaneous change in bearing $d\theta$ is
		$$\frac{d\theta}{dt} = \omega = \frac{v_r - v_l}{l}$$
		and the magnitude of the velocity of point located in centre $c$ of the wheels is simply
		$$\left|v_c\right| = \frac{v_r + v_l}{l}$$
		giving us formulae
		$$\frac{dx}{dt} = \left|v_c\right| \cos{\theta}$$
		$$\frac{dy}{dt} = \left|v_c\right| \sin{\theta}$$
		which we can approximate and discretize into
		$$\delta x = \delta t \left|v_c\right| \cos{\theta}$$
		$$\delta y = \delta t \left|v_c\right| \sin{\theta}$$
		with $v_r$ and $v_l$ determined by
		$$v_{wheel} = sgn(\omega_{wheel}) \frac{\delta E_{wheel}}{\delta t} r_{wheel} $$
		where $E$ is the integer number of encoder ticks on a wheel.

	\subsection{Error Handling}
		\subsubsection{Unable to Retrieve Pet in Loft}
		We will make several attempts at this, adjusting the position of the robot each time. If after a certain number attempts we have not retrieved the pet, we will move on to the next stage.
		\subsubsection{Lost Tape}
		We will adjust proportional, differential, and integral constants to minimize this risk. Additionally, using wheel encoders allows us to have an additional input, which we can use to correct position if the tape is lost.

%--------------------------------------------
%Risk Assessment and Contingency Planning
\newpage
\section{Risk Assessment and Contingency Planning}

%See \autoref{table:Risk Assessment} below:
\begin{table}[ht]
	\caption{Risk Assessment}
	\centering
	\begin{tabular}{ | p{3cm} | p{1.5cm} | p{4cm} | p{5cm} | p{2.4cm} |} 
	\hline
	\textbf{Risk Condition} & \textbf{Level of Risk} & \textbf{Impact to Project} & \textbf{Change to Work Plan} & \textbf{Expected Date of Risk Decision} \\ \hline 
	Unable to attach to zipline	 &  High & Robot can't go back to safety zone & Find other ways to attach to zipline & End of June \\ \hline
	Pets fall off robot arm as it is being picked up &  Medium & Experiment with different magnets strength or find alternative ways to pick up pet & Find other ways to attach to zipline & End of June \\ \hline
	Pets stack on-top of each other and fall off basket & Very Low & Unable to save pets & Change shape of basket & Mid July \\ \hline
	Unable to detect IR Beacon &  Low & Robot may not reach zipline very reliably& Move in a straight line after the tape is finished, until it hits the box. It should be closer to the IR Beacon. OR the robot can drive back to safety zone & Beginning of July \\ \hline
	Unable to properly follow tape & Low & Unable to complete the challenge & Test following tape in a variety of lighting conditions & End of June \\ \hline
	Robot tilts too much as it slides down zipline & Medium & Unable to save several  pets & Adjust shape/size of basket carrying the pets, or change position of zipline arm & End of July \\ \hline
	Unable locate pet on loft & Low & Unable to pick up pet &Experiment with using several attempts to adjust location and lowering robot arm & Mid July \\ \hline
	Run over pet on middle of path & Low & Unable to collect that pet, cause problems in tape-following and picking up other pets & Adjust height and material metal piece used to pick up that pet &Beginning of July \\ \hline
	\end{tabular}

	\label{table:Risk Assessment}
\end{table}
\clearpage
%--------------------------------------------
%Task Schedule and Responsibilities

\section{Task Schedule, Responsibilities and Major Milestones}

	\subsection{Tasks} 
	The main tasks will be construction of the chassis, retrieval arm and zipline apparatus sub-assemblies; construction of motor control and sensor circuitry; and software construction for each stage as described in \autoref{AppendixC}.

	\subsection{Milestones}
		\subsubsection{Tape Following}
		The first milestone will be reliably tape following. Meeting this milestone will require completion of the chassis sub-assembly, two motor control circuits, wheel encoder and tape following QRD circuit completion, and the tape follow subsection of the 'Follow/Retrieve' software mode.
		\subsubsection{IR Homing}
		The second milestone will be reliably following the IR beacon.  This milestone will require the completion of the first milestone as well as completion of the IR circuits and the Beacon Homing software mode.
		\subsubsection{Functional Retrieval Arm}
		The third milestone will be a standalone arm that is capable of picking up and dropping off pets. This milestone will require the completion of the retrieval arm sub-assembly, the motor control circuit, microswitch circuits, and the retrieve subsection of the 'Follow/Retrieve' software mode.
		\subsubsection{Pet Retrieval}
		The fourth milestone will be our robot running through the course and retrieving pets 1-5. This milestone will require the completion of milestones 1-3, the pet marker tape QRD circuitry, and both the 'Follow/Retrieve' and 'Beacon Homing' software modes.
		\subsubsection{Rubble Excavation}
		The fifth milestone will be successfully retrieving pet 6. This will require completion of milestones 1-4 and the 'Ruble Excavation' software mode.
		\subsubsection{Functional Zipline}
		This milestone will be a standalone zipline assembly that can attach to the zipline. This can be accomplished in parallel with milestones 3 and 4. This milestone will require the completion of the zipline apparatus sub-assembly, motor control circuits for the winch and delivery system, and the swing and winch sections of the 'Zipline/Return' software mode.
		\subsubsection{Zipline Use}
		This milestone will be our robot attaching to and using the zipline. This will require the completion of the chassis assembly and drive systems, the 'Functional Zipline' milestone, the side IR sensing circuit, and the 'Zipline/Return' software mode.
		\subsubsection{Complete Course}
		This milestone will be our robot successfully completing the entire course. This will require the completion of all other milestones and the integration of all sub-assemblies and software modes.
		
\begin{table}[ht]
	\caption{Task Schedule}
	\centering
	\begin{tabular}{ | p{1.3cm} | p{3.5cm} | p{3cm} | p{3cm} | p{3.5cm} |} 
	\hline
	\textbf{Week} & \textbf{FJORDBOTTEN} & \textbf{LIM} & \textbf{LIU} & \textbf{ZHANG} \\ \hline 
	7& Chassis & Chassis \newline Drive & Drive & Position Correction \newline Retrieval Arm \\ \hline
	8& Retrieval Arm & Sensors \newline Arm Circuit & Sensors \newline Arm Circuit & Retrieval Arm \newline Retrieval Code \\ \hline
	9& Zipline & Zipline Code \newline Testing Retrieval & Other Circuits & Zipline \newline Rubble Excavation \\ \hline
	10& Testing \newline Rebuilding? & Testing \newline Rebuilding? & Testing \newline Rebuilding? & Testing \newline Rebuilding? \newline Dead Reckoning? \\ \hline
	11 \newline 12 \newline 13 &
	\multicolumn{4}{l|}{\parbox{10cm}{Shared Goals: \newline Retrieves pets on ground \newline Retrieves pet on rafters \newline Retrieves pet in rubble \newline Can zipline back intact \newline Integrate dead reckoning into tape following}} \\ \hline
	\end{tabular}
	\label{table:Task Schedule}
\end{table}

\clearpage
\input{"appendix.tex"}

\end{document}  
