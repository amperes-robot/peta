%----------------------------------------------------------------------------------------
%	PACKAGES AND OTHER DOCUMENT CONFIGURATIONS
%----------------------------------------------------------------------------------------

\documentclass[11pt, oneside]{article} % The default font size and one-sided printing (no margin offsets)
\usepackage{graphicx}
\usepackage{caption}
\usepackage{subcaption}
\graphicspath{{Figures/}}
\usepackage[square, numbers, comma, sort&compress]{natbib} % Use the natbib reference package - read up on this to edit the reference style; if you want text (e.g. Smith et al., 2012) for the in-text references (instead of numbers), remove 'numbers' 
\usepackage[parfill]{parskip}
\usepackage[margin=1in]{geometry}
\usepackage{tabularx}
\usepackage{array}
\usepackage{hyperref}
\hypersetup{
	colorlinks,
	citecolor=blue,
	filecolor=blue,
	linkcolor=blue,
	urlcolor=blue
}
\newcolumntype{P}[1]{>{\centering\arraybackslash}p{#1}}
\renewcommand{\arraystretch}{1.5}
\newcommand{\robotname}{NAME}

\begin{document}

% Define the page headers using the FancyHdr package and set up for one-sided printing

\newcommand{\HRule}{\rule{\linewidth}{0.5mm}} % New command to make the lines in the title page

%----------------------------------------------------------------------------------------
%	TITLE PAGE
%----------------------------------------------------------------------------------------

\begin{titlepage}
\begin{center}

\textsc{\LARGE ENPH 253 Design Proposal}\\[1.5cm]

\HRule \\[0.4cm] % Horizontal line
{\huge \bfseries \robotname}\\[0.07cm]
\HRule \\[1.5cm] % Horizontal line

\begin{minipage}[t]{0.4\textwidth}
\begin{flushleft} \large
Team Members: \\
{Scott Fjordbotten \newline Anne Lim \newline Johnson Liu \newline Kevin Zhang} % Author names
\end{flushleft}
\end{minipage}
\begin{minipage}[t]{0.4\textwidth}
\begin{flushright} \large
\emph{Instructors:} \\
{Dr. Andre Marziali \\ Dr. Jon Nakane \\ Mr. Bernhard Zender \\ Ms. Pamela Rogalski } % Instructors names
\end{flushright}
\end{minipage}\\[3cm]
 
 
{\large June 2015}\\[4cm]
%\includegraphics{Logo} % University/department logo - uncomment to place it
 
\vfill
\end{center}

\end{titlepage}


%----------------------------------------------------------------------------------------
%	Executive Summary
%----------------------------------------------------------------------------------------
\section*{Executive Summary}
This document contains the proposed specifications of a robot designed to complete the 2015 ENPH 253 Pet-Rescue Bots challenge. The robot should be able to navigate through an 8' by 8' foot ramp by following a tape path, and collect ``Pets'', which are Beanie toys of height 6" with magnets attached to them. The robot will then transport itself and the toys to a "safe area" via a zip line of height 19", which goes from the end of the path to the start.

The robot must be able to go through a doorway which is 14" wide and 18" tall. As for the Pets, some will be placed along and beside the path, and others will be located in containers. All will be unattached from the playing surface.

Our robot will be controlled by a TINAH Board using a ATMega128 processor. The proposed design for the robot has an upper bound of 5~kg and a lower bound of 3~kg mass. The design has a length of 15.74", a width of 11.61", and a height of 14.5" when fully retracted.

\newpage

%----------------------------------------------------------------------------------------
%	LIST OF CONTENTS/FIGURES/TABLES PAGES
%----------------------------------------------------------------------------------------

\tableofcontents % Write out the Table of Contents
\listoffigures % Write out the List of Figures
\listoftables % Write out the List of Tables

%----------------------------------------------------------------------------------------

\newpage
%	Document Content
%---------------------------------------------------------------------------------------

%--------------------------------------------
%PREFACE
\section{Preface}

This report was written in collaboration by Kevin Zhang, Scott Fjordbotten, Johnson Liu, and Anne Lim, in the hope of identifying and addressing potential issues with the design. Work was begun on this document in the beginning of June 2015. We would like to thank the teaching assistants and professors of ENPH 253 for their support and advice throughout.

In the process of preparing this proposal, all major design decisions were agreed upon by unanimous consent. Listed below is a summary of the work done by each team member on this report.

\begin{center}
	\begin{tabular}{ l l }
		FJORDBOTTEN & Typesetting, Mechanical Design \\
		LIM & Letter of Transmittal, Editing \\
		LIU & Electronics and Sensors \\
		ZHANG & Software, Editing \\
	\end{tabular}
\end{center}

%--------------------------------------------
%Overview of Basic Strategy
\section{Overview of Basic Strategy}

The robot will contain a meshed in area at the front for carrying the payloads. Pets will be picked up by a plastic arm with a steel bracket on the end which will move along a circular path in the plane parallel to the front of the robot to magnetically attach and collect the pets. Below the steel bracket will be attached a hinged aluminium plate that will be sandwiched between the steel plate and any attached magnet, triggering a micro-switch when a magnet is attached. A rod fixed to the chassis will be located at the end of the arm's path such that the load will be sheared off the metallic plate in order to land in the meshed area.

The robot will start following the tape in the starting location and follow it using a single QRD sensor connected to an analog input on the TINAH board and a proportional-integral-derivative controller. Upon detecting tape markers perpendicular to the main tape path using a side QRD connected to a LM311 comparator and a digital input, the retrieval arm will then be lowered until the attachment signal is received from the micro-switch. At this point, the arm will retract until the load is sheared off using the fixed rod, which can be detected by the falling signal from the micro-switch. The pet located in the middle of the path will magnetically attach to a steel bracket at the front of the robot and will remain attached to this bracket for the duration of the heat. The pet located on the elevated rafters will be located using the IR beacon and the rotary encoders on the wheels to determine the distance travelled since the end of the tape.

After collecting the elevated pet, the robot will continue to follow the IR beacon until the intensity from the beacon reaches a certain threshold, to be determined empirically. Based on encoder data, the robot will pivot 90 degrees to position the retrieval arm along the edge of the box. The retrieval arm will be used to push the Styrofoam rubble aside and retrieve the last pet.

At this point, the robot should be positioned below and facing away from the zipline. An arm located on the left side of the robot opposite the pet retrieval arm and containing a magnetically attached zipline trolley will swing in the plane parallel to the side of the robot to attach the trolley to the zipline. A winch in the centre of the robot and attached to the bottom side of the trolley will simultaneously detach the trolley from the zipline arm and lift the robot off the ground. Once off the ground, the robot will roll down the zipline to the Safe Area by virtue of gravity.

%--------------------------------------------
%Mechanical Design
\section{Mechanical Design}

Our robot will consist of four main mechanical sub-assemblies: The Chassis, The Retrieval Arm, and The Zipline Apparatus. These assemblies have been designed such that they are independent modules that can be built and tested at a basic level in parallel. This design section has been broken into sections to show each element individually before being combined into the complete version of the robot.

	\subsection{The Chassis}

	The Chassis is the base for the robot. The Chassis consists of the Chassis Plate and two supports for the rear drive motors and axles. The Chassis Plate includes two vertical sections at the rear that support the components of the Board Storage Space. These sections will be discussed in more detail in the 'Board Storage Space' subsection. Circular holes have been integrated in the front portion of the chassis plate to allow for wire routing to the tape following and tape detection QRD1114 reflectance sensors. The Chassis Plate and motor supports will be manufactured from 20 gauge aluminium to limit the overall weight of the robot. Additional brackets may be fabricated to reinforce the edge flange joints if necessary. The complete Chassis and Board Storage module is shown in \autoref{fig:chassisAssy}.

	\begin{figure}[h]
		\centering
		\includegraphics[scale=0.5]{"Figures/Chassis Body".jpg}
		\caption[Chassis]{The Chassis}
		\label{fig:1}
	\end{figure}

		\subsubsection{Pet 4 Pickup Arm}
		The fourth pet (in the middle of the path) will be picked up statically. To accomplish this there will be a steel bracket mounted to the front of the chassis via a acrylic arm (See \autoref{fig:front arm}). The steel bracket will be at the height of the pets head and reinforced with magnets to ensure the pet does not fall off in the IR and zipline phases of the course.
	
		\subsubsection{The Sled}
		Rather than building structures to support wheels at the front of the robot, we will construct a sled and attach it to the front of the chassis via two axle mounted brackets (See \autoref{fig:front sled}). The sled will be free to pivot about the mounting axles to minimize the chance of the sled catching on something and forcing the robot off the desired trajectory. The sled will also have a support (not shown) and hole for the tape following QRD. Both the sled and its mounting brackets will be constructed from 20 gauge aluminum.
		
		\subsubsection{Board Storage Space}

		The Board Storage Space is integrated into the rear of the chassis. The vertical sections of the chassis contain slots and tabs to support the three shelves in the Board Storage Space (See \autoref{fig:3}). The shelves will be 3.4" deep and 7.8" long and will have a flange at the front to prevent boards from sliding off the shelves. The two bottom shelves will house circuit boards; once the bottom two shelves have been loaded with boards, the back plate will be screwed on to secure the shelves. The top shelf will support the TINAH in conjunction with the back plate. Two mounting features for the TINAH are on the front of the top shelf while the remaining two are on the top of the back plate.
			
		\begin{figure}[!ht]
			\centering
			\begin{minipage}[t]{.5\textwidth}
				\centering
				\includegraphics[scale=0.5]{"Figures/Board Storage Space".jpg}
				\caption[Board Storage Space]{Board Storage Space}
				\label{fig:2}
			\end{minipage}%
			\begin{minipage}[t]{.5\textwidth}
				\centering
				\includegraphics[scale=0.5]{"Figures/chassis back features".jpg}
				\caption[Board Storage Support Features]{Board Storage Support Features}
				\label{fig:3}
			\end{minipage}
		\end{figure}
		
		\begin{figure}[!ht]
			\begin{minipage}[t]{.5\textwidth}
				\centering
				\includegraphics[scale=0.75]{"Figures/frontSled".jpg}
				\caption[The Sled]{The font sled}
				\label{fig:front sled}
			\end{minipage}%
			\begin{minipage}[t]{.75\textwidth}
				\centering
				\includegraphics[scale=1]{"Figures/frontArm".jpg}
				\caption[Front Arm]{The front arm for statically \\ collecting pet 4}
				\label{fig:front arm}
			\end{minipage}
		\end{figure}
		
		\begin{figure}[!ht]
			\centering
			\includegraphics[scale=0.5]{"Figures/chassisAssy".jpg}
			\caption[Complete Chassis and Board Storage Module]{Complete Chassis and Board Storage Module}
			\label{fig:chassisAssy}
		\end{figure}
		
	\subsection{The Retrieval Arm}
	The retrieval arm consists of two main section: the supporting body (\autoref{fig:armBody}) and the arm end (\autoref{fig:armEnd}). The supporting body is constructed out of 20 gauge aluminum. It houses the arm sections, gears and pet knock off tool (\autoref{fig:knockOffTool})and will attach to the chassis via pop rivet. There will be two arm sections, one strait and one that will double as the large driving gear. Both arms will be cut out of acrylic(\autoref{fig:arm}). The gear ratio between the driving DC motor and the arm gear will be 5:1 as determined in \autoref{AppendixA}. The knock of tool will be made of 3.2mm diameter round steel and positioned so that pets will be knocked off the arm above the chassis. The length and center of rotation of the arm were chosen such that the arm is the same distance from the chassis center(eight inches) when the pick up surface is at both six and twelve inches from the playing surface so that both arms can be used to pick up the pets placed on the surface and the pet that is elevated.
	
		\subsubsection{Retrieval Arm End}
		The arm end will consist of a body made of acrylic, L-brackets made of steel to magnetically attach to pets, and aluminum sheets that will be pushed into micro-switches when a pet is picked up(\autoref{fig:armEnd}). There will be two such L-bracket assemblies. The smaller assembly, facing the front of the robot, will be used to pick up pets 1-4 and 5. This arm will align with the knock off tool to deposit pets in the chassis. The larger assembly, facing the rear of the robot, is at the bottom of the body and will be used to collect pet 6 from the box of foam. This assembly will be reinforced with magnets and the 6th pet will remain on the bracket for the remainder of the course.
		
	\begin{figure}[h]
		\centering
		\begin{subfigure}[b]{0.4\textwidth}
			\centering
			\includegraphics[scale=0.5]{"Figures/armBodyFront".jpg}
			\caption{Front View}
		\end{subfigure}
		~
		\begin{subfigure}[b]{0.5\textwidth}
			\centering
			\includegraphics[scale=0.5]{"Figures/armBodyRear".jpg}
			\caption{Rear View}
		\end{subfigure}
		\caption[Retrieval Arm Module]{Retrieval Arm Module}
		\label{fig:armBody}
	\end{figure}
		
	\begin{figure}[h]
		\centering
		\begin{minipage}[t]{.5\textwidth}
			\centering
			\includegraphics[scale = 0.4]{"Figures/armEnd".jpg}
			\caption[Retrieval Arm End]{Retrieval Arm End}
			\label{fig:armEnd}
		\end{minipage}%
		\begin{minipage}[t]{.5\textwidth}
			\centering
			\includegraphics[scale=0.4]{"Figures/armArms".jpg}
			\caption[Retrieval Arm Arms]{Retrieval Arm Arms}
			\label{fig:arm}
		\end{minipage}
	\end{figure}
	
	\begin{figure}[!h]
		\centering
		\includegraphics[scale=0.4]{"Figures/knockOffTool".jpg}
		\caption[Knock Off Tool]{Knock Off Tool}
		\label{fig:knockOffTool}
	\end{figure}
	
	\subsection{The Zipline Apparatus}
	The zipline apparatus is contained in a body similar to the pet retrieval arm body. This body houses the motors for the zipline delivery system and the winch system, the gears for these systems the zipline delivery arm and the zipline trolley (\autoref{fig:zipModule}). The body will be made from 20 gauge aluminum and the gears will be made ot of acrylic.
	
		\subsubsection{Zipline Delivery System}
		The zipline delivery system consists of a guide that swings the zipline trolley into place. The guide has flanges around all sides to prevent the trolley from falling out before the 
		trolley is deployed. The trolley frame and guide will be made of aluminum and will have small magnets attached to prevent premature separation.  
		
		\subsubsection{The Zipline Trolley}
		The zipline trolley consists of a frame, winch attachment axle and roller (\autoref{fig:zipTrolley}). The frame was designed with flanges all the way around to improve rigidity. A basic FEA was carried out in Solidworks. The results showed stresses much below the plastic limit and deflections of a fraction of a millimeter (See \autoref{fig:zipFEA}). Although the validity of the FEA is questionable, the results would have to be off by two orders of magnitude for deflections to be concerning so we are confident our design is sufficient. The winch belt will be attached to the trolley via a pin at the bottom of the frame. The pin bracket will be manufactured from 20 gauge aluminum. The top of trolley supports the roller which will sit on the zipline, allowing the robot to use the force of gravity to slide to safety once lifted by the winch. The roller will be lathed out of Ultra High Molecular Weight Polyethylene.
		
		\subsubsection{The Winch System}
		The robot will be lifted off the ground by the winch system. The winch will be driven by a worm gear system with a 40:1 gear ratio to ensure ample torque and prevent back driving (\autoref{fig:winch}). The winch will be connected to the trolley by a 1.5" wide fabric (exact material to be determined likely similar to a backpack adjustment strap). 
		
		\newpage
		
		\begin{figure}[h]
			\centering
			\begin{minipage}[t]{.5\textwidth}
				\centering
				\includegraphics[scale = 0.4]{"Figures/ziplineModule".jpg}
				\caption[Zipline Module]{Zipline Module}
				\label{fig:zipModule}
			\end{minipage}%
			\begin{minipage}[t]{.5\textwidth}
				\centering
				\includegraphics[scale=0.4]{"Figures/zipTrolley".jpg}
				\caption[Zipline Trolley]{Zipline Trolley}
				\label{fig:zipTrolley}
			\end{minipage}
		\end{figure}	
		
		\begin{figure}[h]
			\centering
			\begin{minipage}[t]{.5\textwidth}
				\centering
				\includegraphics[scale = 0.4]{"Figures/ziplineZipperFEA".jpg}
				\caption[Zipline Trolley FEA]{Zipline Trolley FEA}
				\label{fig:zipFEA}
			\end{minipage}%
			\begin{minipage}[t]{.5\textwidth}
				\centering
				\includegraphics[scale=0.4]{"Figures/winch".jpg}
				\caption[WinchSystem]{Winch System}
				\label{fig:winch}
			\end{minipage}
		\end{figure}
			
%--------------------------------------------
%Drive and Actuator Systems
\section{Drive and Actuator Systems}


TEST
	\subsection{Drive System}
	
	\subsection{Retrieval Arm Actuation}
	
	\subsection{Zipline Apparatus Actuation}
	
		\subsubsection{Zipline Delivery System Actuation}
		
		\subsubsection{Zipline Winch Actuation}


%--------------------------------------------
%Sensor System

\section{Sensor System}
	
	\subsection{Tape Following}
	
	There will be 1 QRD1114 phototransistor at the front of the robot for reading reflectance values to detect tape on the playing surface. It will work together with wheel encoder to ensure efficient tape following. Another QRD1114 will be placed on the side of the robot for detecting tapes leading to the pets. 
	
	\subsection{IR Detection}
	There will be 2 QSD124 IR phototransistors for detecting the 10kHz rescue beacon. By analyzing the change in values read from these 2 IR phototransistors while the robot is moving, the distance and direction to the 10kHz rescue beacon can be known. 
	\subsection{Wheel Encoders}
	To detect the pet on the elevated rafter, a wheel encoder will be hard coded to determine if the robot has reached the location of it. Upon detection, the pet will be held in front of the robot for the rest of the heat. 
	\subsection{Pet Pick-up Detection}
	A microswitch status is associated with whether a pet has been knocked into the container by the fixed rod.
%--------------------------------------------
%Electrical Design

\section{Electrical Design}

	\subsection{Driving and Arm Actuation}
	
	There will be two motors for driving, each powering one wheel using differential steering. Each motor will be connected to a motor output on TINAH via an external H-bridge with a comparator attached. Similarly, an H-bridge is also used for actuating the arm for picking up pets and another for actuating the zipline delivery apparatus. The circuit schematic is in \hyperref[sch:A]{Schematic A} of \hyperref[appendixB]{Appendix B}.
	
	\subsection{Detecting Pets and Following Tape}
	
	There will be a QRD1114 photo-transistor for tape following by PID algorithm. It will be connected to an analog input pin on TINAH. Another QRD1114 will be placed on the side of the robot and connected to a digital input on TINAH with a comparator for detecting tapes leading to pets. The circuit for these 2 photo-transistors is in \hyperref[sch:B]{Schematic B} of \hyperref[appendixB]{Appendix B}. 
	
	\subsection{IR Beacon Detection}
	
	\hyperref[sch:C]{Schematic C} of \hyperref[appendixB]{Appendix B} section shows the circuit for the 2 IR detectors used for detecting the IR beacons during the competition. The type of IR detector we'll be using is QSD124. 
	
	\subsection{Winch Actuation}
	
	For actuating the winch which will move the roller of the zipline onto the pipe, \hyperref[sch:D]{Schematic D} of \hyperref[appendixB]{Appendix B} will be used. It consists mainly of an N-type MOSFET switch for driving the motor for winch in one direction. 
	
	\subsection{Pet Pick-Up Detection}
	
	The micro-switch in \hyperref[sch:E]{Schematic D} of \hyperref[appendixB]{Appendix B} indicates how to detect if a pet has been picked up and placed into the container. If the value read is high, it means no pet is attached to the retrieval arm. If the value read is low, then it means there is a pet attached to the retrieval arm. 
	
	
	\begin{table}[h]
		\caption{Table of TINAH Pin Connections}
		\centering
		\begin{tabular}[t]{|P{0.5in}|p{2in}|}
			\hline
			\textbf{Digital Pin} & \textbf{Input/Output} \\
			\hline
			0 & Pet Marker QRD \\
			\hline
			1 & Pickup Microswitch Lower \\
			\hline
			2 & Pickup Microswitch Upper \\
			\hline
			3 & Winch Enable \\
			\hline
			4 & Wheel Encoder Left \\
			\hline
			5 & Wheel Encoder Right\\
			\hline
		\end{tabular}
		\quad
		\begin{tabular}[t]{|P{0.5in}|p{2in}|}
			\hline
			\textbf{Analog Pin} & \textbf{Input/Output} \\
			\hline
			0 & IR Sensor Left \\
			\hline
			1 & IR Sensor Right \\
			\hline
			2 & Tape Follower QRD \\
			\hline
			\multicolumn{2}{c}{ } \\
			\hline
			\textbf{Motor Pin} & \textbf{Output}\\
			\hline
			0 & Drive Motor Left \\
			\hline
			1 & Drive Motor Right \\
			\hline
			2 & Retrieval Arm Motor \\
			\hline
			3 & Zipline Arm Motor \\
			\hline
		\end{tabular}
		\label{table:TINAHpins}
	\end{table}
	
	\begin{table}[h]
		\caption{Table of PCB Information}
		\centering
		\begin{tabular}[t]{ | P{0.7in} | p{1.5in} | P{0.7in} | p{1.2in} | P{1in} | }
			\hline
			\textbf{PCB Number} & \textbf{Purpose} & \textbf{Size (mm)} & \textbf{Components Connected} & \textbf{Rails Needed} \\
			\hline
			1 & H-bridges: Drive Motors Left and Right, Retrieval Arm Motor, Zipline Arm Motor & 180 x 70 & TINAH, 4 DC motors & GND, 5V, 15V\\
			\hline
			2 & IR detection circuitry & X x X & TINAH, 2 photodiodes & GND, 5V, $\pm$ 9V\\
			\hline
			3 & ??? & X x X & ??? & ???\\
			\hline
		\end{tabular}
		\label{table:PCBInfo}
	\end{table}
	
%--------------------------------------------
%Softwate and Code Algorithms

\section{Software and Code Algorithms}	
The software will operate statefully, with one main control loop that handles the primary strategy and execution. At the same time, a 10kHz timer interrupt will run in parallel which will handle the time-sensitive operation of polling input pins.
The main control loop will call upon a variety of different modes that describe the current operation at any given point in the program's execution. A summary of the different modes and their transitions is described in \autoref{fig:4} below.

\begin{figure}[h]
	\centering
	\includegraphics[scale=0.5]{"Figures/Software Modes".jpg}
	\caption[Software Modes]{Software Modes}
	\label{fig:4}
\end{figure}

A more detailed description of the modes is shown in \autoref{table:Software Modes}.

\begin{table}
	\caption{Table of Software Modes}
	\centering
	\begin{tabular}{ | c | p{5in} | }
	\hline
	\textbf{Mode} & \textbf{Description} \\ \hline
	Main Menu & This is the mode automatically entered when the board boots. The knobs are used to cycle between menu options. Additionally, any mode can be cancelled to return to the Main Menu. \\ \hline
	Options Menu & Allows runtime parameters to be modified using the second knob. \\ \hline
	Segment Select & Starts the selected mode from a list containing modes during program execution. \\ \hline
	Course Begin & Transitions to the Follow/Retrieval mode. \\ \hline
	Follow/Retrieval & Follows the tape and picks up the first three animals. Upon reaching the end of the tape (once all horizontal marks have been detected), switches to the Beacon Homing mode. \\ \hline
	Beacon Homing & Using the two forward IR sensors, navigates towards the IR beacon. Picks up the elevated pet on the rafters by measuring distance from the end of the tape. Switches to Rubble Excavation upon IR sensor intensity reaching a threshold to be determined empirically. \\ \hline
	Rubble Excavation & Turns 90 degrees and brushes off the top layer of foam using the arm. Attempts to find and retrieve the pet buried in the foam, and upon retrieving it (or on approaching the 2 minute time limit), switches to Zipline/Return. \\ \hline
	Zipline/Return &
	Uses the side IR sensor to align the robot with the beacon, raises the zipline arm, and cranks the winch up to return to the Safe Area. Returns to Main Menu afterwards.
	\\ \hline
	\end{tabular}
	\label{table:Software Modes}
\end{table}

	\subsection{I/O}
	
	On the 10kHz timer interrupt will be attached a procedure that handles pin inputs. Because the default \texttt{analogRead} function blocks until the ADC conversion finishes (approximately a 1000-cycle process), carrying out the conversion in parallel can significantly decrease loop latency. Additionally, polling digital pin inputs on the timer interrupt can provide a better guarantee that changes in digital inputs will not be missed because of latency in the main loop.


\begin{figure}[h]
	\centering
	\includegraphics[scale=0.5]{"Figures/Software State Diag".jpg}
	\caption[Software State Diagram]{Software State Diagram}
	\label{fig:SoftStates}
\end{figure}

	\subsection{Error Handling}
		\subsubsection{Unable to Retrieve Pet in Loft}
		We will make several attempts at this, adjusting the position of the robot each time. If after a certain number attempts we have not retrieved the pet, we will move on to the next stage.
		\subsubsection{Lost Tape}
		We will adjust proportional, differential, and integral constants to minimize this risk. Additionally, using wheel encoders allows us to have an additional input, which we can use to correct position if the tape is lost.

%--------------------------------------------
%Risk Assessment and Contingency Planning
\section{Risk Assessment and Contingency Planning}

See \autoref{table:Risk Assessment} below:
\begin{table}[h]
	\caption{Risk Assessment}
	\centering
	\begin{tabular}{ | p{3cm} | p{2cm} | p{4cm} | p{4cm} | p{3cm} |} 
	\hline
	\textbf{Risk Condition} & \textbf{Likelihood} & \textbf{Impact to Project} & \textbf{Change to Work Plan} & \textbf{Expected Date of Risk Decision} \\ \hline 
	Unable to attach to zipline	 &  High & Robot can?t go back to safety zone & Find other ways to attach to zipline & End of June \\ \hline
	Pets fall off robot arm as it is being picked up &  Medium & Experiment with different magnets strength or find alternative ways to pick up pet & Find other ways to attach to zipline & End of June \\ \hline
	Pets stack on-top of each other and fall off basket & Very Low & Unable to save pets & Change shape of basket & Mid July \\ \hline
	Unable to detect IR Beacon &  Low & Robot may not reach zipline very reliably& Move in a straight line after the tape is finished, until it hits the box. It should be closer to the IR Beacon. OR the robot can drive back to safety zone & Beginning of July \\ \hline
	Unable to properly follow tape & Low & Unable to complete the challenge & Test following tape in a variety of lighting conditions & End of June \\ \hline
	Robot tilts too much as it slides down zipline & Medium & Unable to save several  pets & Adjust shape/size of basket carrying the pets, or change position of zipline arm & End of July \\ \hline
	Unable locate pet on loft & Low & Unable to pick up pet &Experiment with using several attempts to adjust location and lowering robot arm & Mid July \\ \hline
	Run over pet on middle of path & Low & Unable to collect that pet, cause problems in tape-following and picking up other pets & Adjust height and material metal piece used to pick up that pet &Beginning of July \\ \hline
	\end{tabular}

	\label{table:Risk Assessment}
\end{table}
\clearpage
%--------------------------------------------
%Task Schedule and Responsibilities

\section{Task Schedule and Responsibilities}

\begin{table}[ht]
	\caption{Task Schedule}
	\centering
	\begin{tabular}{ | p{4.5cm} | p{3cm} | p{3cm} | p{3cm} | p{3cm} |} 
	\hline
	\textbf{Week} & \textbf{Anne} & \textbf{Kevin} & \textbf{Johnson} & \textbf{Scott} \\ \hline 
	7& & & & \\ \hline
	8& & & & \\ \hline
	9& & & & \\ \hline
	10& & & & \\ \hline
	11 (Time Trials)& & & & \\ \hline
	12 (Team Presentations)& & & & \\ \hline
	13 (Competition Week)& & & & \\ \hline
	\end{tabular}
	\label{table:Task Schedule}
\end{table}

\clearpage
\input{"appendix.tex"}

\end{document}  
